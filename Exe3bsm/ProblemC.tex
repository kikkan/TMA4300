% Options for packages loaded elsewhere
\PassOptionsToPackage{unicode}{hyperref}
\PassOptionsToPackage{hyphens}{url}
\PassOptionsToPackage{dvipsnames,svgnames,x11names}{xcolor}
%
\documentclass[
]{article}
\title{Title}
\usepackage{etoolbox}
\makeatletter
\providecommand{\subtitle}[1]{% add subtitle to \maketitle
  \apptocmd{\@title}{\par {\large #1 \par}}{}{}
}
\makeatother
\subtitle{Course}
\author{Christian Oppegård Moen}
\date{DD MM YYYY}

\usepackage{amsmath,amssymb}
\usepackage{lmodern}
\usepackage{iftex}
\ifPDFTeX
  \usepackage[T1]{fontenc}
  \usepackage[utf8]{inputenc}
  \usepackage{textcomp} % provide euro and other symbols
\else % if luatex or xetex
  \usepackage{unicode-math}
  \defaultfontfeatures{Scale=MatchLowercase}
  \defaultfontfeatures[\rmfamily]{Ligatures=TeX,Scale=1}
\fi
% Use upquote if available, for straight quotes in verbatim environments
\IfFileExists{upquote.sty}{\usepackage{upquote}}{}
\IfFileExists{microtype.sty}{% use microtype if available
  \usepackage[]{microtype}
  \UseMicrotypeSet[protrusion]{basicmath} % disable protrusion for tt fonts
}{}
\makeatletter
\@ifundefined{KOMAClassName}{% if non-KOMA class
  \IfFileExists{parskip.sty}{%
    \usepackage{parskip}
  }{% else
    \setlength{\parindent}{0pt}
    \setlength{\parskip}{6pt plus 2pt minus 1pt}}
}{% if KOMA class
  \KOMAoptions{parskip=half}}
\makeatother
\usepackage{xcolor}
\IfFileExists{xurl.sty}{\usepackage{xurl}}{} % add URL line breaks if available
\IfFileExists{bookmark.sty}{\usepackage{bookmark}}{\usepackage{hyperref}}
\hypersetup{
  pdftitle={Title},
  pdfauthor={Christian Oppegård Moen},
  colorlinks=true,
  linkcolor={Maroon},
  filecolor={Maroon},
  citecolor={Blue},
  urlcolor={blue},
  pdfcreator={LaTeX via pandoc}}
\urlstyle{same} % disable monospaced font for URLs
\usepackage[margin=1in]{geometry}
\usepackage{color}
\usepackage{fancyvrb}
\newcommand{\VerbBar}{|}
\newcommand{\VERB}{\Verb[commandchars=\\\{\}]}
\DefineVerbatimEnvironment{Highlighting}{Verbatim}{commandchars=\\\{\}}
% Add ',fontsize=\small' for more characters per line
\usepackage{framed}
\definecolor{shadecolor}{RGB}{248,248,248}
\newenvironment{Shaded}{\begin{snugshade}}{\end{snugshade}}
\newcommand{\AlertTok}[1]{\textcolor[rgb]{0.94,0.16,0.16}{#1}}
\newcommand{\AnnotationTok}[1]{\textcolor[rgb]{0.56,0.35,0.01}{\textbf{\textit{#1}}}}
\newcommand{\AttributeTok}[1]{\textcolor[rgb]{0.77,0.63,0.00}{#1}}
\newcommand{\BaseNTok}[1]{\textcolor[rgb]{0.00,0.00,0.81}{#1}}
\newcommand{\BuiltInTok}[1]{#1}
\newcommand{\CharTok}[1]{\textcolor[rgb]{0.31,0.60,0.02}{#1}}
\newcommand{\CommentTok}[1]{\textcolor[rgb]{0.56,0.35,0.01}{\textit{#1}}}
\newcommand{\CommentVarTok}[1]{\textcolor[rgb]{0.56,0.35,0.01}{\textbf{\textit{#1}}}}
\newcommand{\ConstantTok}[1]{\textcolor[rgb]{0.00,0.00,0.00}{#1}}
\newcommand{\ControlFlowTok}[1]{\textcolor[rgb]{0.13,0.29,0.53}{\textbf{#1}}}
\newcommand{\DataTypeTok}[1]{\textcolor[rgb]{0.13,0.29,0.53}{#1}}
\newcommand{\DecValTok}[1]{\textcolor[rgb]{0.00,0.00,0.81}{#1}}
\newcommand{\DocumentationTok}[1]{\textcolor[rgb]{0.56,0.35,0.01}{\textbf{\textit{#1}}}}
\newcommand{\ErrorTok}[1]{\textcolor[rgb]{0.64,0.00,0.00}{\textbf{#1}}}
\newcommand{\ExtensionTok}[1]{#1}
\newcommand{\FloatTok}[1]{\textcolor[rgb]{0.00,0.00,0.81}{#1}}
\newcommand{\FunctionTok}[1]{\textcolor[rgb]{0.00,0.00,0.00}{#1}}
\newcommand{\ImportTok}[1]{#1}
\newcommand{\InformationTok}[1]{\textcolor[rgb]{0.56,0.35,0.01}{\textbf{\textit{#1}}}}
\newcommand{\KeywordTok}[1]{\textcolor[rgb]{0.13,0.29,0.53}{\textbf{#1}}}
\newcommand{\NormalTok}[1]{#1}
\newcommand{\OperatorTok}[1]{\textcolor[rgb]{0.81,0.36,0.00}{\textbf{#1}}}
\newcommand{\OtherTok}[1]{\textcolor[rgb]{0.56,0.35,0.01}{#1}}
\newcommand{\PreprocessorTok}[1]{\textcolor[rgb]{0.56,0.35,0.01}{\textit{#1}}}
\newcommand{\RegionMarkerTok}[1]{#1}
\newcommand{\SpecialCharTok}[1]{\textcolor[rgb]{0.00,0.00,0.00}{#1}}
\newcommand{\SpecialStringTok}[1]{\textcolor[rgb]{0.31,0.60,0.02}{#1}}
\newcommand{\StringTok}[1]{\textcolor[rgb]{0.31,0.60,0.02}{#1}}
\newcommand{\VariableTok}[1]{\textcolor[rgb]{0.00,0.00,0.00}{#1}}
\newcommand{\VerbatimStringTok}[1]{\textcolor[rgb]{0.31,0.60,0.02}{#1}}
\newcommand{\WarningTok}[1]{\textcolor[rgb]{0.56,0.35,0.01}{\textbf{\textit{#1}}}}
\usepackage{longtable,booktabs,array}
\usepackage{calc} % for calculating minipage widths
% Correct order of tables after \paragraph or \subparagraph
\usepackage{etoolbox}
\makeatletter
\patchcmd\longtable{\par}{\if@noskipsec\mbox{}\fi\par}{}{}
\makeatother
% Allow footnotes in longtable head/foot
\IfFileExists{footnotehyper.sty}{\usepackage{footnotehyper}}{\usepackage{footnote}}
\makesavenoteenv{longtable}
\usepackage{graphicx}
\makeatletter
\def\maxwidth{\ifdim\Gin@nat@width>\linewidth\linewidth\else\Gin@nat@width\fi}
\def\maxheight{\ifdim\Gin@nat@height>\textheight\textheight\else\Gin@nat@height\fi}
\makeatother
% Scale images if necessary, so that they will not overflow the page
% margins by default, and it is still possible to overwrite the defaults
% using explicit options in \includegraphics[width, height, ...]{}
\setkeys{Gin}{width=\maxwidth,height=\maxheight,keepaspectratio}
% Set default figure placement to htbp
\makeatletter
\def\fps@figure{htbp}
\makeatother
\setlength{\emergencystretch}{3em} % prevent overfull lines
\providecommand{\tightlist}{%
  \setlength{\itemsep}{0pt}\setlength{\parskip}{0pt}}
\setcounter{secnumdepth}{-\maxdimen} % remove section numbering
\usepackage[width=0.8\textwidth]{caption}
\ifLuaTeX
  \usepackage{selnolig}  % disable illegal ligatures
\fi

\begin{document}
\maketitle

{
\hypersetup{linkcolor=}
\setcounter{tocdepth}{3}
\tableofcontents
}
\begin{Shaded}
\begin{Highlighting}[]
\FunctionTok{source}\NormalTok{(}\StringTok{"./additionalFiles/probAhelp.R"}\NormalTok{)}
\FunctionTok{source}\NormalTok{(}\StringTok{"./additionalFiles/probAdata.R"}\NormalTok{)}
\NormalTok{figPath }\OtherTok{=} \StringTok{"./Figures/"}
\end{Highlighting}
\end{Shaded}

\hypertarget{problem-c-the-em-algorithm-and-bootstrapping}{%
\section{Problem C: The EM-algorithm and bootstrapping}\label{problem-c-the-em-algorithm-and-bootstrapping}}

Let \(x_1,...x_n\) and \(y_1,...,y_n\) be independet random variables, where
\[
x_i \sim \text{Exp}(\lambda_0) \ \ \text{and} \ \ y_i \sim \text{Exp}(\lambda_1)
\]

We observe

\[
z_i =\text{max}(x_i, y_i) \ \ \text{for} \ \ i=1,...,n
\]

and

\[
u_i=I(x_i \geq y_i) \ \ \text{for} \ \ i=1,...,n.
\]

\hypertarget{section}{%
\subsection{1.}\label{section}}

The joint distribution of \((x_i, y_i),i=1,..n\) is given by

\[
f(x, y | \lambda_0, \lambda_1)=\prod_{i=1}^{n} f_{x}(x_i | \lambda_0) \cdot f_{y}(y_i|\lambda_1)
\]
\[
=\prod_{i=1}^{n}\lambda_0 e^{-\lambda_0 x_i} \cdot \lambda_1 e^{-\lambda_1 y_i}.
\]
This means that the log likelihood is given by
\[
\ln f(x,y|\lambda_0, \lambda_1)=\sum_{i=1}^{n} \ln \lambda_0+ \ln \lambda_1-\lambda_0 x_i - \lambda_1 y_i=n(\ln \lambda_0+ \ln \lambda_1)-\lambda_0\sum_{i=1}^n x_i -\lambda_1\sum_{i=1}^n y_i \quad.
\]
We want to find
\[
E\left[ \ln f(x,y|\lambda_0, \lambda_1)| z, u, \lambda_0^{(t)}, \lambda_1^{(t)} \right].
\]

which is given by

\[
Q(\lambda_0, \lambda_1 | \lambda_0^{(t)}, \lambda_1^{(t)})=E \left[\sum_{i=1}^{n}n(\ln \lambda_0+ \ln \lambda_1)-\lambda_0\sum_{i=1}^n x_i -\lambda_1\sum_{i=1}^n y_i \ \ | \ \ z,u,\lambda_0^{(t)}, \lambda_1^{(t)} \right]
\]
\[
=-n(\ln\lambda_0+\ln\lambda_1) -\lambda_0\sum_{i=1}^n E(x_i\mid z_i,u_i,\lambda_0^{(t)},\lambda_1^{(t)}) -\lambda_1\sum_{i=1}^n E(y_i\mid z_i,u_i,\lambda_0^{(t)},\lambda_1^{(t)}).
\]
Now, we want to find \(E(x_i\mid z_i,u_i,\lambda_0^{(t)},\lambda_1^{(t)})\) and \(E(y_i\mid z_i,u_i,\lambda_0^{(t)},\lambda_1^{(t)}).\) We start by considering the first conditional expectation. This can found by first considering
\[
\begin{aligned}
  f(x_i\mid z_i,u_i,\lambda_0^{(t)},\lambda_1^{(t)})  
  &= \begin{cases} z_i \quad \quad \quad \quad \quad \text{for} & u_i=1 \\ \frac{\lambda_0^{(t)}\exp(-\lambda_0^{(t)}x_i)}{1-\exp(-\lambda_0^{(t)}z_i)} \ \ \text{for} & u_i=0 \end{cases} \quad.
\end{aligned}
\]
The expectation is given by

\[
E[x_i|z_i, u_i, \lambda_0^{(t)}, \lambda_1^{(t)}]=u_i z_i + (1-u_i)\int_{0}^{z_i} x_i \frac{\lambda_0^{(t)}\exp(-\lambda_0^{(t)}x_i)}{1-\exp(-\lambda_0^{(t)}z_i)} dx_i
\]
where

\[
\int_{0}^{z_i} x_i \frac{\lambda_0^{(t)}\exp(-\lambda_0^{(t)}x_i)}{1-\exp(-\lambda_0^{(t)}z_i)} dx_i=
\left [- \frac{(\lambda_0^{(t)}x+1)\exp(-\lambda_0^{(t)} \cdot (x_i-z_i))}{\lambda_0^{(t)}(1-\exp(\lambda_0 z))} \right]_{0}^{z_i}
\]
\[
=\frac{\exp(\lambda_0^{(t)} z_i)- \lambda_0 z_i-1}{\lambda_0^{(t)} \cdot (\exp(\lambda_0^{(t)}-1))} \implies
\]
\[
E[x_i|z_i, u_i, \lambda_0^{(t)}, \lambda_1^{(t)}]=u_i z_i+(1-u_i) \cdot \frac{\exp(\lambda_0^{(t)} z_i)- \lambda_0 z_i-1}{\lambda_0^{(t)} \cdot (\exp(\lambda_0^{(t)}-1))}.
\]
We also need to find \(E[y_i|z_i, u_i, \lambda_0^{(t)}, \lambda_1^{(t)}]\). We first consider the pdf

\[
\begin{aligned}
  f(y_i\mid z_i,u_i,\lambda_0^{(t)},\lambda_1^{(t)})  
  &= \begin{cases} z_i \quad \quad \quad \quad \quad \text{for} & u_i=1 \\ \frac{\lambda_1^{(t)}\exp(-\lambda_1^{(t)}y_i)}{1-\exp(-\lambda_1^{(t)}z_i)} \ \ \text{for} & u_i=0 \end{cases} \quad.
\end{aligned}
\]
The conditional expectation is given by

\[
E[y_i|z_i, u_i, \lambda_0^{(t)}, \lambda_1^{(t)}]=(1 - u_i) z_i + u_i \int_{0}^{z_i} y_i \frac{\lambda_1^{(t)} \exp\left\{-\lambda_1^{(t)} y_i\right\}}{1 - \exp\left\{-\lambda_1^{(t)} z_i\right\}} dy_i 
\]
where

\$\$
\int\_\{0\}\^{}\{z\_i\} y\_i \frac{\lambda_1^{(t)}\exp(-\lambda_1^{(t)}y_i)}{1-\exp(-\lambda_1^{(t)}z_i)} dy\_i=

\frac{\exp({\lambda_1^{(t)} z_i})-\lambda_1^{(t)}-1}{\lambda_1^{(t)}(\exp(\lambda_1^{(t) }z_i)-1)}

\$\$
\#\# 2.

In this problem we want to implement the EM-algorithm. We have found the conditional expectation \(Q(\lambda_0,\lambda_1)=Q(\lambda_0, \lambda_1 | \lambda_0^{(t)}, \lambda_1^{(t)}).\) This corresponds to the E-step in the EM algorithm. In the M-step of the algorithm is to determine

\[
(\lambda_0^{(t+1)}, \lambda_1^{(t+1)})=\text{argmax} \ \ Q(\lambda_0, \lambda_1).
\]

This can be found be finding the partial derivates and \(Q(\lambda_0, \lambda_1)\) and set them equal to zero.

\[
\frac{\partial}{\partial \lambda_0} Q(\lambda_0, \lambda_1)=
\frac{n}{\lambda_0}-\sum_{i=1}^{n} \left ( u_i z_i+(1-u_i) \left ( \frac{1}{\lambda_0^{(t)}}-\frac{z_i}{e^{\lambda_0^{(t)}z_i}-1} \right ) \right)=0
\]

\[
\frac{\partial}{\partial \lambda_1} Q(\lambda_0, \lambda_1)=
\frac{n}{\lambda_1}-\sum_{i=1}^{n} \left ( u_i z_i+(1-u_i) \left ( \frac{1}{\lambda_1^{(t)}}-\frac{z_i}{e^{\lambda_1^{(t)}z_i}-1} \right ) \right)=0
\]

We solve these two equations for \(\lambda_0\) and \(\lambda_1\) respectively. This gives the M-step

\[
\lambda_0^{(t+1)}=n/\sum_{i=1}^{n} \left ( u_i z_i+(1-u_i) \left ( \frac{1}{\lambda_0^{(t)}}-\frac{z_i}{e^{\lambda_0^{(t)}z_i}-1} \right ) \right)
\]
\[
\lambda_1^{(t+1)}=n/\sum_{i=1}^{n} \left ( u_i z_i+(1-u_i) \left ( \frac{1}{\lambda_1^{(t)}}-\frac{z_i}{e^{\lambda_1^{(t)}z_i}-1} \right ) \right)
\]
Let \(\lambda^{(t)}=(\lambda_0^{(t)}, \lambda_1^{(t)}).\)
We want to implement the EM-algorithm and we use the convergence criterion

\[
d(x^{(t+1)}, x^{t})= || {\lambda}^{(t+1)} - {\lambda}^{(t)}||_2<\epsilon.
\]

The function below returns the conditional expectation, that is the E-step of the EM algorithm.

\begin{Shaded}
\begin{Highlighting}[]
\NormalTok{cond\_expectation }\OtherTok{\textless{}{-}} \ControlFlowTok{function}\NormalTok{(lambda0, lambda1, lambda0t, lambda1t, u, z) \{}
\NormalTok{    n }\OtherTok{=} \FunctionTok{length}\NormalTok{(u)}
\NormalTok{    exp }\OtherTok{=}\NormalTok{ n }\SpecialCharTok{*}\NormalTok{ (}\FunctionTok{log}\NormalTok{(lambda0) }\SpecialCharTok{+} \FunctionTok{log}\NormalTok{(lambda1)) }\SpecialCharTok{{-}}\NormalTok{ (lambda0 }\SpecialCharTok{*} \FunctionTok{sum}\NormalTok{(u }\SpecialCharTok{*}\NormalTok{ z }\SpecialCharTok{+}\NormalTok{ (}\DecValTok{1} \SpecialCharTok{{-}}\NormalTok{ u) }\SpecialCharTok{*}\NormalTok{ (}\DecValTok{1}\SpecialCharTok{/}\NormalTok{lambda0t }\SpecialCharTok{{-}}
\NormalTok{        (z)}\SpecialCharTok{/}\NormalTok{(}\FunctionTok{exp}\NormalTok{(lambda0t }\SpecialCharTok{*}\NormalTok{ z) }\SpecialCharTok{{-}} \DecValTok{1}\NormalTok{)))) }\SpecialCharTok{{-}}\NormalTok{ (lambda1 }\SpecialCharTok{*} \FunctionTok{sum}\NormalTok{(u }\SpecialCharTok{*}\NormalTok{ z }\SpecialCharTok{+}\NormalTok{ (}\DecValTok{1} \SpecialCharTok{{-}}\NormalTok{ u) }\SpecialCharTok{*}\NormalTok{ (}\DecValTok{1}\SpecialCharTok{/}\NormalTok{lambda1t }\SpecialCharTok{{-}}
\NormalTok{        (z)}\SpecialCharTok{/}\NormalTok{(}\FunctionTok{exp}\NormalTok{(lambda1t }\SpecialCharTok{*}\NormalTok{ z) }\SpecialCharTok{{-}} \DecValTok{1}\NormalTok{))))}
    \FunctionTok{return}\NormalTok{(exp)}
\NormalTok{\}}
\end{Highlighting}
\end{Shaded}

Under is a function that implement M-step.

\begin{Shaded}
\begin{Highlighting}[]
\NormalTok{M\_step }\OtherTok{\textless{}{-}} \ControlFlowTok{function}\NormalTok{(lambda0t, lambda1t, u, z) \{}
\NormalTok{    lambda0next }\OtherTok{=}\NormalTok{ n}\SpecialCharTok{/}\NormalTok{(}\FunctionTok{sum}\NormalTok{(u }\SpecialCharTok{*}\NormalTok{ z }\SpecialCharTok{+}\NormalTok{ (}\DecValTok{1} \SpecialCharTok{{-}}\NormalTok{ u) }\SpecialCharTok{*}\NormalTok{ (}\DecValTok{1}\SpecialCharTok{/}\NormalTok{lambda0t }\SpecialCharTok{{-}}\NormalTok{ (z)}\SpecialCharTok{/}\NormalTok{(}\FunctionTok{exp}\NormalTok{(lambda0t }\SpecialCharTok{*}\NormalTok{ z) }\SpecialCharTok{{-}}
        \DecValTok{1}\NormalTok{))))}
\NormalTok{    lambda1next }\OtherTok{=}\NormalTok{ n}\SpecialCharTok{/}\NormalTok{(}\FunctionTok{sum}\NormalTok{(u }\SpecialCharTok{*}\NormalTok{ z }\SpecialCharTok{+}\NormalTok{ (}\DecValTok{1} \SpecialCharTok{{-}}\NormalTok{ u) }\SpecialCharTok{*}\NormalTok{ (}\DecValTok{1}\SpecialCharTok{/}\NormalTok{lambda1t }\SpecialCharTok{{-}}\NormalTok{ (z)}\SpecialCharTok{/}\NormalTok{(}\FunctionTok{exp}\NormalTok{(lambda1t }\SpecialCharTok{*}\NormalTok{ z) }\SpecialCharTok{{-}}
        \DecValTok{1}\NormalTok{))))}

    \FunctionTok{return}\NormalTok{(}\FunctionTok{list}\NormalTok{(}\AttributeTok{lambda0 =}\NormalTok{ lambda0next, }\AttributeTok{lambda1 =}\NormalTok{ lambda1next))}
\NormalTok{\}}
\end{Highlighting}
\end{Shaded}

Under the ME algorithm is implemented

\begin{Shaded}
\begin{Highlighting}[]
\NormalTok{ME\_algorithm }\OtherTok{\textless{}{-}} \ControlFlowTok{function}\NormalTok{(lambda, u, z, }\AttributeTok{epsilon =} \FloatTok{1e{-}04}\NormalTok{) \{}
\NormalTok{    i }\OtherTok{=} \DecValTok{0}
\NormalTok{    lambda0 }\OtherTok{=}\NormalTok{ lambda[}\DecValTok{0}\NormalTok{]}
\NormalTok{    lambda1 }\OtherTok{=}\NormalTok{ lambda[}\DecValTok{1}\NormalTok{]}
    \ControlFlowTok{while}\NormalTok{ (norm }\SpecialCharTok{\textless{}}\NormalTok{ epsilon }\SpecialCharTok{\&\&}\NormalTok{ i }\SpecialCharTok{\textless{}} \DecValTok{500}\NormalTok{) \{}
        \CommentTok{\# E{-}step}
\NormalTok{        Q }\OtherTok{=} \FunctionTok{cond\_expectation}\NormalTok{(lambda0, lambda1, )}
        \CommentTok{\# M{-}step}

\NormalTok{    \}}
\NormalTok{\}}
\end{Highlighting}
\end{Shaded}

\begin{Shaded}
\begin{Highlighting}[]
\NormalTok{l }\OtherTok{\textless{}{-}} \FunctionTok{numeric}\NormalTok{(}\DecValTok{2}\NormalTok{)}
\NormalTok{l}
\end{Highlighting}
\end{Shaded}

\begin{verbatim}
## [1] 0 0
\end{verbatim}

\hypertarget{section-1}{%
\subsection{4.}\label{section-1}}

We want to find an analytical formula for \(f_{Z_i,U_i}(z_i, u_i| \lambda_0, \lambda_1).\)

\[
f_{Z_i,U_i}(z_i, u_i | \lambda_0, \lambda_1)=P(\text{max}(X_i, Y_i)=z , I(X_i \geq Y_i)=u_i | \lambda_0, \lambda_1)
\]
\[
=u_i P(\text{max}(X_i, Y_i)=z_i, X)
\]

\end{document}
